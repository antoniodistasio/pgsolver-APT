\section{The Generic Solver}

Solving a parity game is done by a central module called the \emph{generic solver}.
It combines the universal optimisations described above with any of the implemented
algorithms of heuristics described above. This is realised by taking a solver, i.e.\
one of these algorithms or heuristics as a parameter. Then it roughly works as follows.
\begin{enumerate}
\item Self-cycles are eliminated from the game, and attractors of nodes for which the
      self-cycle is part of a winning strategy are computed and removed. 
\item The entire game is decomposed into SCCs.
\item Terminal SCCs are solved as follows.
      \begin{enumerate}
        \item Priorities are compressed.
        \item The SCC is checked for being a special case of a game.
              \begin{itemize}
                 \item If it is, winning regions and strategies are constructed accordingly.
                 \item Otherwise, the solver given as the parameter is used to solve this
                       SCC.
              \end{itemize}
        \item Attractors of the computed winning regions are also computed, together with 
              corresponding strategies which are added to the winning regions and strategies.
        \item The computed winning regions are removed from the game.
        \item All non-terminal SCCs which have lost some nodes in the removal of these
              attractors are again decomposed into more fine-grained SCCs.
      \end{enumerate}
\item Step 3 is repeated until the entire game is solved.
\end{enumerate}
Note that the removal of terminal SCCs will make other, previously non-terminal SCCs terminal.
Also, note that this scheme is sound -- the regions and strategies it computes are
in fact winning regions and winning strategies for the corresponding players on the given
game -- if the parameter solver is sound. Hence, it can also be used with sound heuristics.
Furthermore, it is complete if it is guaranteed that the parameter solver solves at least 
one node of every SCC that it is given. Hence, the solvers used as backends need not be 
complete for the generic solver to be complete. This is why this scheme can solve whole games
using heuristics that are incomplete themselves.

We also note that the features SCC decomposition, detection of special cases, and priority
compression can be switched off via command-line options. This may be useful when the performance 
of an algorithm on its own is to be measured. In addition, it is possible to turn the feature 
priority propagation on in which case it is done before priority compression. However, in general 
this does not seem to be an optimisation since it slows down virtually any backend.  



%%% Local Variables:
%%% mode: latex
%%% TeX-master: "main"
%%% End:
